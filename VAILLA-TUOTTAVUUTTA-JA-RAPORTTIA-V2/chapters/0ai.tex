\section*{\Large \textbf{Tekoälyn käyttö tässä työssä}}

Tässä työssä käytetyt tekoälypohjaiset sovellukset on esitetty seuraavassa taulukossa:

\begin{center}
    \begin{tabularx}{\linewidth}{X|l}
        \toprule
        \textbf{Sovellus} & \textbf{Versio} \\
        \midrule
        OpenAI ChatGPT & GPT-5 \\
        Google Gemini & 2.5 \\
        \bottomrule
    \end{tabularx}
\end{center}

\section*{Tekoälyn käytön tarkoitus}

Tekoälytyökaluja on hyödynnetty raportin laadinnan tukena. Niitä käytettiin muun muassa tekstin ideointiin ja muokkaamiseen, rakenteen suunnitteluun, kieliasun parantamiseen sekä lähdeviitteiden ja kuvien tuottamiseen. Tekoäly nopeutti työskentelyä ja tuki selkeän ja johdonmukaisen raportin rakentamista.

\section*{Tekoälyn käyttö työn eri osissa}

Tekoälyä on hyödynnetty laajasti eri vaiheissa: lähteiden kartoittamisessa, tiedon jäsentämisessä, raportin rakenteen suunnittelussa, tekstin kirjoittamisessa ja viimeistelyssä sekä visualisointien luomisessa. Hankkeen teeman mukaisesti tekoälyllä on ollut vahva ja keskeinen rooli koko prosessin ajan.
