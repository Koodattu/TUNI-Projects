\chapter{Johtopäätökset ja suositukset}

Tämä luku muodostaa tutkimusraportin päätösluvun, jossa analysoitu tieto muutetaan konkreettisiksi toimenpiteiksi. Se kokoaa yhteen aiempien lukujen (luku 2: tekoälyn mahdollisuudet ja sovellusalueet; luku 3: Etelä-Pohjanmaan alueellinen ja toimialakohtainen potentiaali; luku 4: käyttöönoton haasteet ja ratkaisumallit) syvälliset havainnot. Luvun ensisijaisena tehtävänä on esittää raportin keskeiset johtopäätökset sekä tarjota käytännönläheisiä suosituksia eteläpohjalaisille pk-yrityksille tekoälyn hyödyntämiseen ja alueen kehittäjille niiden tukemiseen. Näin tämä luku vastaa raportin viimeisiin tutkimuskysymyksiin, erityisesti Etelä-Pohjanmaan alueellisiin johtopäätöksiin ja käytännön suosituksiin, syventäen ja syntetisoiden edellisten lukujen tietoa kohti tulevaisuuden toimintaa.

\section{Suositukset eteläpohjalaisille pk-yrityksille}

Aiemmissa luvuissa (luku 2 ja luku 3) on käsitelty tekoälyn tuomia merkittäviä mahdollisuuksia ja sovellusalueita sekä luvussa 4 käyttöönoton keskeisiä haasteita, kuten osaamisen, resurssien ja datan puutetta. Nämä haasteet huomioiden ja tekoälyn täyden potentiaalin hyödyntämiseksi eteläpohjalaisilla pk-yrityksillä on edessään strateginen ja jatkuva kehityspolku. Seuraavat suositukset on laadittu ohjaamaan tätä matkaa kohti systemaattisempaa ja tuloksellisempaa tekoälyn hyödyntämistä:

\begin{enumerate}
    \item \textbf{Määrittele selkeät tavoitteet ja strategia:} Tekoälyn käyttöönoton tulee olla integroitu osaksi yrityksen liiketoimintastrategiaa. Asettakaa sekä lyhyen että pitkän aikavälin tavoitteet, jotka tukevat laajempaa strategista implementointia ja varmistavat, että tekoäly palvelee todellisia liiketoimintatarpeita \cite{Jantti2025}. Kuten luvussa 4 korostettiin, ylimmän johdon sitoutuminen on kriittistä, eikä tekoälyn käyttöönottoa tule jättää yksinomaan IT-vastaavan vastuulle. Sen tulee olla koko organisaation yhteinen muutosmatka, jossa tekninen ja substanssiosaaminen kohtaavat parhaiden ideoiden tuottamiseksi \cite{Jantti2025, LahtinenHumala2023}.

    \item \textbf{Edistä vaiheittaista käyttöönottoa ja kokeilukulttuuria:} Aloittakaa tekoälyn hyödyntäminen vaiheittain, keskittyen ensin valmiisiin ratkaisuihin ja yksittäisiin käyttötapauksiin \cite{Jantti2025}. Kokeilkaa aktiivisesti ilmaisia tai edullisia generatiivisen tekoälyn palveluita pienimuotoisesti ilman suuria alkuinvestointeja. Tämä matalan kynnyksen lähestymistapa auttaa ratkaisemaan luvussa 4 tunnistettuja kustannushaasteita, madaltaa kokeilun riskiä ja edistää organisaation oppimista ja sopeutumista uuteen teknologiaan \cite{KauttonenAsikainen2024, Krapi2024}. Verkostoitukaa aktiivisesti kollegoiden, muiden yritysten ja tekoälyasiantuntijoiden kanssa jakaaksenne kokemuksia ja parhaita käytäntöjä.

    \item \textbf{Panosta osaamisen kehittämiseen:} Tekoälyn täyden potentiaalin saavuttaminen edellyttää jatkuvia investointeja työntekijöiden koulutukseen ja riskienhallintaan \cite{Chaudhuri2022}. Kuten luvussa 4 todettiin, osaamisen puute on merkittävin este tekoälyn käyttöönotolle. Edistäkää jatkuvaa kehotemuotoilutaitojen kehittämistä ja tarjotkaa työntekijöille mahdollisuuksia ammatilliseen kehitykseen sekä itseohjautuvaan oppimiseen \cite{Ilonen2025, Lahtinen2024, Chaudhuri2022}. Tämä auttaa kuromaan umpeen osaamisvajetta ja lisää työntekijöiden luottamusta uuteen teknologiaan.

    \item \textbf{Datan laadun ja hallinnan merkitys:} Datan laatu ja saatavuus ovat tiettyyn käyttöön räätälöidyn tekoälyn käyttöönoton kriittisiä edellytyksiä, kuten luvussa 4 tuotiin esille. Generatiivisen tekoälyn avulla tehtävät datalähtöiset analyysit ja tiedolla johtaminen vaativat, että tekoälylle syötetty tieto ei sisällä virheitä tai ole harhaanjohtavaa. Pk-yritysten tulisi pyrkiä datakeskeisyyteen poistamalla dataan ja prosesseihin liittyvät siilot ja keskittämällä IT-arkkitehtuurinsa tekoälyn tehokkaamman hyödyntämisen varmistamiseksi \cite{KauttonenAsikainen2024, Alnajjar2025}. Pienemmille yrityksille kustannustehokkain lähestymistapa on hyödyntää valmiita tekoälymenetelmiä oman, korkealaatuisen datansa kanssa, sillä tämä soveltuu parhaiten rajallisille resursseille \cite{Alnajjar2025}.

    \item \textbf{Ylimmän johdon sitoutuminen ja eettisyys:} Kuten \textbf{luvussa 4} painotettiin, tekoälyn onnistunut käyttöönotto edellyttää johdon vahvaa sitoutumista ja organisaation valmiutta \cite{Chaudhuri2022}. Johdon tulee varmistaa, että tekoälyn käyttö on turvallista ja eettisten periaatteiden mukaista. Laatikaa kirjalliset ohjeistukset generatiivisen tekoälyn käyttöön, korostaen, ettei arkaluonteisia tietoja syötetä tekoälymalleihin ja että tuotettu sisältö on aina oikoluettava ja tarkastettava \cite{Kang2024, Ilonen2025}. Tämä vastaa luvussa 4 tunnistettuihin tietoturva- ja etiikkahuoliin ja varmistaa vastuullisen käyttöönoton.

    \item \textbf{Hyödynnä ulkoista tukea ja verkostoja:} Älkää epäröikö hakea ulkopuolista apua, erityisesti osaamisen puutteen ja resurssirajoitteiden vuoksi, kuten luvussa 4 kuvattiin. Hyödyntäkää esimerkiksi korkeakoulujen lopputyöprojekteja tai European Digital Innovation Hubs (EDIH) -verkostoja, kuten Finnish AI Region (FAIR) \cite{Jantti2025, Seppala2023, Asikainen2025}. Yhteistyö yliopistojen, julkisten tutkimusorganisaatioiden ja muiden asiantuntijaorganisaatioiden kanssa tarjoaa arvokasta asiantuntemusta ja resursseja, jotka voivat täydentää sisäistä osaamista ja nopeuttaa käyttöönottoa \cite{OECD2025}.

    \item \textbf{Priorisoi sovelluskohteet:} Tunnistakaa ne alueet, joilla tekoäly voi tuottaa suurimman hyödyn ja parantaa toimintaa konkreettisesti \cite{Khan2025}. Aloittakaa kokeilut toistuvista ja rutiininomaisista tehtävistä, joissa tekoäly voi vapauttaa työntekijöiden aikaa monimutkaisempiin ja arvoa tuottavampiin tehtäviin. Tämä tehostaa toimintaa ja luo konkreettisia tuottavuushyötyjä, kuten luvussa 2 esitettiin, esimerkiksi automatisoimalla rutiininomaisia ajojärjestelytehtäviä tai kysynnän ennustamista logistiikka-alalla \cite{Ruusunen2025}.

    \item \textbf{Panosta luottamukseen ja käyttäjäkokemukseen:} Käyttäjäkokemuksen helppous, hyödyllisyys ja personoitujen kokemusten tarjoaminen ovat avainasemassa tekoälypalveluiden hyväksynnän lisäämisessä \cite{Kang2024}. Luottamukseen perustuvan käyttökokemuksen luominen parantamalla palvelun turvallisuutta, tietosuojaa ja läpinäkyvyyttä on kriittistä tekoälyn laajemman hyväksynnän ja muutosvastarinnan lieventämisen kannalta, kuten luvussa 4 todettiin \cite{Kang2024}.

\end{enumerate}

\begin{figure}[h]
\centering
\includegraphics[width=8cm]{images/suspri.png}
\caption[Suositusten priorisointi]{Suositusten 2×2-matriisi vaikutuksen ja toteutuksen vaivan mukaan: nopeat voitot, strategiset hankkeet, helpot kokeilut ja pitkän aikavälin investoinnit.}
\label{fig:suositukset}
\end{figure}

\section{Suositukset alueen kehittäjille ja tukiorganisaatioille}

Kuten luvussa 4 yksityiskohtaisesti käsiteltiin, pk-yritykset kohtaavat tekoälyn käyttöönotossa useita systeemisiä esteitä, kuten osaamisen puutetta, rahoitushaasteita ja sääntelyepävarmuutta. Etelä-Pohjanmaan alueellisten kehittäjien, julkisen sektorin ja tukiorganisaatioiden rooli on kriittinen näiden esteiden ylittämisessä ja suotuisan toimintaympäristön luomisessa. Seuraavat politiikkasuositukset auttavat tukemaan alueen pk-yrityksiä tekoälymatkalla ja varmistamaan alueen kilpailukyvyn:

\begin{enumerate}
    \item \textbf{Kohdennettu koulutus ja osaamisen kehittäminen:} Osaamisen puute on merkittävin este tekoälyn käyttöönotolle pk-yrityksissä \cite{Ohlsbom2025}, \cite{Yrittajat2025}. Alueellisten toimijoiden tulisi tukea räätälöityjen koulutusohjelmien ja päivitettävien pätevyyskehysten luomista, jotta yritykset voivat tunnistaa ja kehittää tarvitsemaansa osaamista \cite{OECD2025}. Erityisesti verkkokurssit, webinaarit ja opetusvideot ovat yrittäjien suosimia koulutusmuotoja \cite{Yrittajat2025}. Nämä investoinnit auttavat työntekijöitä työskentelemään uuden teknologian kanssa ja parantavat heidän tuottavuuttaan sen sijaan, että heidät syrjäytettäisiin \cite{Thelle2024}.

    \item \textbf{Julkisen datan saatavuuden ja laadun parantaminen:} Datan laatu ja saatavuus ovat kriittisiä tekoälyn onnistuneelle käyttöönotolle \cite{Alnajjar2025, OECD2025}. Alueellisten ja kansallisten toimijoiden tulee panostaa julkisten tietovarantojen laatuun ja saavutettavuuteen. Yritykset raportoivat, että julkisen datan hankintaprosessit ovat monimutkaisia ja data vanhentunutta, mikä hidastaa datalähtöistä päätöksentekoa \cite{OECD2025}.

    \item \textbf{Sääntely ja vastuukysymykset:} Epäselvät säännöt ja tietoturvaan liittyvät huolet hidastavat tekoälyn käyttöönottoa. Pk-yritykset tarvitsevat selkeää ohjeistusta siitä, kuka on vastuussa, jos tekoälyn käytössä tapahtuu virheitä \cite{OECD2025}. Sääntelyä tulisi päivittää, että se helpottaa tekoälyn käyttöönottoa ja takaa tietosuojan toteutumisen \cite{Alnajjar2025}.

    \item \textbf{Tukimekanismit ja rahoitusmahdollisuudet:} Rahoituksen saatavuus on edelleen haaste. Vain 21\% pk-yrityksistä on tietoisia hallituksen digitalisaatiotuista, ja vain 10\% on hyötynyt niistä \cite{BianchiniSancho2025}.

    \item \textbf{Ekosysteemiyhteistyö:} Tekoälyn mahdollisuuksiin tarttumisessa tarvitaan laajoja ekosysteemejä, kuten tutkimus-, innovaatio- ja liiketoimintaekosysteemejä, joissa eri toimijoilla on erilaisia rooleja \cite{Seppala2023}. European Digital Innovation Hubs (EDIH), kuten Finnish AI Region (FAIR), edistävät julkisen ja yksityisen sektorin yhteistyötä tarjotakseen asiantuntemusta ja tukea pk-yritysten tekoälymatkalle ja madaltaakseen kokeilukynnystä \cite{Asikainen2025, TEMAI40_2021}. Lisäksi Innokaupungit toimivat alueellisina innovaatioekosysteemialoitteina. Tämä vastaa \textbf{luvussa 4} korostettuun yhteistyön ja verkostoitumisen tarpeeseen. Tällaiset ekosysteemit auttavat pk-yrityksiä ylittämään osaamisvajeen ja integroitumaan tehokkaammin digitaaliseen murrokseen.
    
    \item \textbf{Suomen kansainvälinen asema:} Suomen tulee vahvistaa asemaansa kansainvälisenä suunnannäyttäjänä kaksoissiirtymässä ja kannustaa yrityksiä vahvistamaan rooliaan ja vaikuttavuuttaan EU-tason päätöksenteossa, TKI-hankkeissa ja verkostoissa \cite{Tyoe_elinkeinoministerio2022}. Visiossa korostetaan vastuullista tekoälykehitystä ja organisaatioiden muutoskyvykkyyttä \cite{BusinessFinland2025}. Lisäksi Suomen tulee panostaa osaamisen kehittämiseen sekä tietoliikenne- ja laskentainfrastruktuurin vahvistamiseen, jotta tekoälyn ja muiden kärkiteknologioiden hyödyntäminen voi edetä kestävällä tavalla \cite{Tyoe_elinkeinoministerio2022}. Tämä on tärkeää, jotta Suomi pysyy tekoälyn kehityksen eturintamassa ja voi hyödyntää kansainvälistä yhteistyötä pk-yritysten hyväksi.
    
    \item \textbf{Kansalliset kielimallit:} Julkisen sektorin kokeilut generatiivisen tekoälyn kanssa lainvalmistelussa osoittivat suomenkielisen koulutusaineiston puutteen ja kielimallien pienen konteksti-ikkunan haasteina, korostaen tarvetta panostaa kansallisten kielimallien kehittämiseen \cite{BusinessFinland2025}. Tämä on kriittistä, jotta suomalaiset pk-yritykset voivat hyödyntää tekoälyä tehokkaasti omissa prosesseissaan ja ylittää kielestä johtuvia haasteita, jotka liittyvät \textbf{luvussa 4} käsiteltyyn datan saatavuuteen ja laatuun.

\end{enumerate}

\section{Yhteenveto ja tulevaisuuden näkymät}

Tämä raportti on syventynyt tekoälyn (erityisesti generatiivisen tekoälyn) mahdollisuuksiin, haasteisiin ja käyttöönoton nykytilaan suomalaisissa pk-yrityksissä, keskittyen Etelä-Pohjanmaan alueellisiin ja toimialakohtaisiin näkökulmiin. Vaikka valtaosa pk-yrityksistä ei vielä hyödynnä tekoälyä aktiivisesti, kiinnostus ja käyttöaikeet ovat selvästi kasvussa \cite{Yrittajat2025}, kuten luvuissa 1 ja 3 vahvistettiin. Osaamisen kehittäminen ja teknologian tehokas hyödyntäminen ovatkin elintärkeitä pk-yritysten kilpailukyvylle vuoteen 2030 mennessä, jolloin tekoälyasiantuntijuudesta on todennäköisesti tullut kiinteä osa monen yrityksen henkilöstörakennetta \cite{LahtinenHumala2023}.

Raportin keskeinen havainto on, että eroavaisuudet tekoälyn hyödyntämisessä syventävät kilpailukyvyn kuilua edelläkävijöiden ja muiden yritysten välillä. Kuten luvussa 3 todettiin, Etelä-Pohjanmaan pk-yrityksissä tekoälyn hyödyt rajoittuvat usein vielä yksittäisten toimintojen, kuten markkinoinnin, tehostamiseen. Teknologian käyttöönoton laiminlyönti voi kuitenkin johtaa jälkeenjäämiseen toimialan uudistuessa. Tämä voi pidemmällä aikavälillä johtaa nykyisten liiketoimintamallien vanhenemiseen, kun tekoälystä muodostuu kilpailukyvyn välttämättömyys \cite{Jantti2025}. Konsulttiyritys Implement Consulting Groupin arvion \cite{Thelle2024} mukaan jo viiden vuoden viivästys generatiivisen tekoälyn hyödyntämisessä voisi pienentää Suomen BKT:n potentiaalista vuosikasvua 20–25 miljardista eurosta 4–5 miljardiin euroon. Havainto korostaa nopeutetun käyttöönoton kriittisyyttä koko Suomen talouden ja siten myös alueiden kilpailukyvyn kannalta.

Positiivista on, kuten luvussa 2 mainittiin, että generatiivinen tekoäly voi auttaa kuromaan umpeen taitovajeita erityisesti heikompien suoriutujien osalta. Useat kokeelliset tutkimukset osoittavat, että juuri vähemmän kokeneet ja heikommin suoriutuvat työntekijät hyötyvät tekoälystä suhteellisesti eniten \cite{Thelle2024, cui2025, brynjolfsson2025, dellacqua2023}. Esimerkiksi eräässä tutkimuksessa keskimääräistä heikommin suoriutuneet konsultit paransivat suorituslaatuaan tekoälyn avulla 43~\%, kun taas parhaimmistoon kuuluvien parannus oli vain 17~\% \cite{dellacqua2023}. Toisessa tutkimuksessa vähiten kokeneiden asiakaspalvelijoiden tuottavuus kasvoi noin 30~\%, ja tekoälyn todettiin myös nopeuttavan uusien työntekijöiden oppimista \cite{brynjolfsson2025}. On kuitenkin tärkeää tiedostaa, että tietyissä tehtävissä, jotka ylittävät tekoälyn nykykyvykkyydet, sen käyttö voi jopa heikentää suorituskykyä \cite{dellacqua2023}. Vaikka generatiiviset tekoälybotit tarjoavat pk-yrityksille tilapäistä kilpailuetua varhaisessa vaiheessa, etu on usein lyhytikäinen, kun kilpailijat omaksuvat samat käytännöt \cite{Heikkila2025}. Siksi luvussa 4 esitetyt jatkuva kehittyminen ja innovointi ovat ensiarvoisen tärkeitä.

Tulevaisuudessa generatiivisen tekoälyn odotetaan muuttavan liiketoimintaa merkittävästi, nostaen automaation tasoa ja yhdistyvän mahdollisesti kvanttilaskennan kaltaisiin teknologioihin \cite{Rezazadeh2025}. Esimerkiksi taloushallinnon alalla asiantuntijuus korostuu, rutiinitehtävät vähenevät ja ihmiselle jää vastuu päätöksenteosta, oppimisesta, vuorovaikuttamisesta sekä monitahoisten tilanteiden käsittelystä \cite{Kuuva2024}. Taloushallinnossa tekoälyä ei nähdä uhkana työpaikoille, vaan pikemminkin tehtäviä monipuolistavana ja haastavana elementtinä \cite{Kuuva2024}.

Yhteenvetona Etelä-Pohjanmaan pk-yritysten menestys tekoälyaikakaudella riippuu niiden kyvystä omaksua uusia teknologioita, panostaa osaamiseen, hyödyntää tukiverkostoja ja ekosysteemejä. Jatkuva oppiminen ja sopeutuminen ovat avainasemassa tässä dynaamisessa ympäristössä, jotta alueen yritykset voivat muuttaa haasteet mahdollisuuksiksi ja varmistaa kilpailukykynsä tulevaisuudessa \cite{Lahtinen2024}. Tämän raportin suositukset pyrkivät tukemaan siirtymää kohti tekoälyä hyödyntävää, innovatiivisempaa ja kilpailukykyisempää Etelä-Pohjanmaata.