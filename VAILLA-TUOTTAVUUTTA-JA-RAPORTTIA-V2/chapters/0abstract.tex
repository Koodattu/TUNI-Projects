Raportti tarkastelee, miten generatiivinen tekoäly voi toimia eteläpohjalaisten pk-yritysten voimavarana. Se kokoaa ajantasaisen kirjallisuuskatsauksen ja alueellisen analyysin. Raportti jäsentää ilmiön käsitteet ja hyödyntämistavat sekä alueen toimialakohtaiset mahdollisuudet teknologia-, valmistava- ja elintarviketeollisuudessa sekä palvelu- ja asiantuntijatyössä. Raportissa käydään läpi myös keskeisiä käyttöönoton esteitä sekä esitetään konkreettisia suosituksia yrityksille ja alueen kehittäjille. Suomen AI-valmius on kansainvälisesti vahva, mutta pk-sektorilla käyttöönotto on yhä epätasaista: kiinnostus kasvaa, silti merkittävä osa yrityksistä ei vielä hyödynnä tekoälyä järjestelmällisesti. Generatiivinen tekoäly tarjoaa matalan kynnyksen tuottavuusloikkia erityisesti sisällöntuotannossa, asiakaspalvelussa, myynnissä ja tietotyössä; teollisuudessa potentiaali konkretisoituu prosessien optimoinnissa, laadunvarmistuksessa ja ennakoivassa kunnossapidossa.

Keskeiset pullonkaulat ovat osaamisvaje, kustannus- ja resurssirajoitteet, datan saatavuus ja laatu, muutosvastarinta, tietoturvaan, eettiseen käyttöön ja sääntelyyn liittyvät huolet, ROI-epävarmuus sekä integraatiohaasteet. Raportti suosittaa vaiheittaista etenemistä valmiilla työkaluilla ja rajatuilla käyttötapauksilla, systemaattista osaamisen kehittämistä (erityisesti kehotemuotoilu ja johtotason AI-lukutaito), datakeskeistä arkkitehtuuria, selkeitä käyttöohjeita ja riskienhallintaa, sekä aktiivista verkostoitumista EDIH-ekosysteemiin (FAIR), korkeakouluihin ja asiantuntijatoimijoihin. Julkiset instrumentit (esim. Business Finland) ja alueelliset palvelut madaltavat kokeilukynnystä. Päätöksentekijöille korostetaan kohdennettua koulutusta, julkisen datan laadun parantamista, sääntelyn selkeyttämistä, rahoitusmekanismien vahvistamista, ekosysteemiyhteistyön syventämistä sekä panostusta kansallisiin kielimalleihin. Johtopäätös on, että tekoälyn nopea ja hallittu käyttöönotto, jatkuva oppiminen ja yhteistyö voivat muodostua merkittäväksi kilpailutekijäksi pk-yrityksille myös Etelä-Pohjanmaalla. Kilpailukyky 2030-luvulla rakentuu kuitenkin monista eri tekijöistä, joista teknologian hyödyntäminen on yksi keskeinen osa.