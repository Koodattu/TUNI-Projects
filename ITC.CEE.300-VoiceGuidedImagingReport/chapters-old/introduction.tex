%!TEX root = ../main.tex
%********************************
\chapter{Introduction}
\section{Background}
Recent advances in generative artificial intelligence (AI) have made it possible to create and edit multimedia content using AI-driven tools. Applications based on generative models are now widely used in areas like content creation, marketing, and design. These technologies often rely on cloud-based services, which provide ease of use and scalability but come with trade-offs in terms of privacy, cost, and reliance on external infrastructure.

This project explores an alternative approach — using self-hosted AI models to build applications that prioritize privacy, flexibility, and cost-efficiency. By hosting models locally, users gain greater control over data handling, performance, and updates. However, local deployment also presents challenges, including hardware requirements, memory optimization, and scalability limitations, especially when working with multiple AI models simultaneously.

A key focus of this project is intention recognition — the ability to interpret user commands and respond with meaningful actions. Intention recognition makes it possible to create interactive and responsive systems that can take natural language inputs, infer actions, and execute tasks with minimal manual intervention.

The Voice Guided Imaging application serves as a proof-of-concept for these ideas. It combines speech recognition, intent processing, and generative AI models to demonstrate how self-hosted AI can handle complex tasks like generating and editing images and videos based on voice commands.

\section{Objectives}
\subsection{Primary Goal}
The primary objective of this project is to explore how to build an AI-powered application using multiple self-hosted generative AI models, focusing specifically on intention recognition. The aim was to test the feasibility of integrating different AI models into a modular pipeline that could process natural language inputs, infer user intentions, and generate output such as images and videos.

\subsection{Secondary Goals}
\subsubsection{Local Hosting Feasibility}
Investigate whether locally hosted AI models can replace cloud-based services for multimedia creation, addressing privacy and performance concerns.

\subsubsection{Pipeline Design}
Learn how to structure modular AI pipelines that combine speech-to-text, intent recognition, and generative AI models into a unified application.

\subsubsection{Performance and Optimization}
Identify challenges in latency, memory usage, and scalability, and evaluate solutions to improve real-time performance.

\subsubsection{Real-World Testing}
Assess how well the system works under practical conditions, including its ability to handle different inputs, languages, and commands.

\section{Scope}
This report focuses on documenting the design and development of the Voice Guided Imaging application. It covers the tools, technologies, and methods used to create the system, as well as the challenges encountered and lessons learned during implementation.

While the application demonstrates the feasibility of using self-hosted AI models, this report does not include detailed comparisons with cloud-based services or performance benchmarks — these topics are reserved for future research and the author’s thesis work. Instead, this report serves as a practical case study on building AI applications using locally hosted models, offering insights into their capabilities and limitations.