\chapter{Nykytila ja potentiaali Etelä-Pohjanmaalla: Toimialakohtainen analyysi}

Tämä luku syventyy tekoälyn nykytilaan ja potentiaaliin Suomen sisällä, keskittyen erityisesti \textbf{Etelä-Pohjanmaan alueen pieniin ja keskisuuriin yrityksiin (pk-yrityksiin)}. Rakentaen edellisen luvun (luku 2) yleiskatsauksen ja käsitteellisen pohjan päälle, luvussa siirrytään kansallisesta ja yleisestä tasosta alueelliseen ja toimialaspesifiin tarkasteluun. Luvun tavoitteena on tunnistaa Etelä-Pohjanmaan yritysten erityispiirteet ja tarjota konkreettisia esimerkkejä tekoälyn soveltamisesta alueen keskeisillä toimialoilla. Tämä analyysi pohjustaa tulevien lukujen haasteiden käsittelyä (luku 4) ja suosituksia (luku 5).

\section{Yleiskuva alueen tilanteesta}

Kuten luvuissa 1 ja 2 todettiin, Suomi on Euroopan unionin kärkimaiden joukossa tekoälyn käyttöönotossa yrityksissä, ja Tilastokeskuksen mukaan keväällä 2024 jo 24 prosenttia suomalaisista yrityksistä käytti tekoälyteknologioita \cite{Tilastokeskus2024}. Kuitenkin, tekoälyn käyttöönotossa on edelleen merkittäviä alueellisia ja yrityskokoon liittyviä eroja. Pk-yritysten osalta luvut ovat matalampia: esimerkiksi vuonna 2023 vain 14 prosenttia suomalaisista pk-yrityksistä oli ottanut käyttöön vähintään yhden tekoälyteknologian \cite{Heikkila2025}.

Vaikka Suomen Yrittäjien selvityksen mukaan noin kolmannes (33 \%) pk-yrityksistä kokee tekoälyn käytön ajankohtaiseksi nyt tai seuraavan \textbf{vuoden} aikana, \textbf{vain 12 prosenttia} käyttää sitä säännöllisesti ja vajaa kolmannes (30 \%) satunnaisesti \cite{Ohlsbom2025}. Merkittävä osa, jopa 60–70 prosenttia, suomalaisista pk-yrityksistä ei vielä hyödynnä tekoälyä lainkaan liiketoiminnassaan \cite{LahtinenHumala2023, Yrittajat2025, HaagaHelia2023}. Tämä osoittaa, että pk-yritysten tekoälymatka on usein vasta alkuvaiheessa, ja ne tarvitsevat tukea osaamisen ja pilottien kehittämisessä, turvautuen usein ulkopuolisiin kumppanuuksiin teknisessä toteutuksessa \cite{Seppala2023, HaagaHelia2023, LahtinenHumala2023}. OECD:n mukaan edistyneiden digitaalisten työkalujen käyttö on yleisempää keskisuurissa ja pienissä yrityksissä kuin mikroyrityksissä tai itsenäisillä ammatinharjoittajilla \cite{BianchiniSancho2025}.

Etelä-Pohjanmaan alueella pk-yritysten tekoälyn käyttöönotto noudattelee pitkälti maan keskiarvoa: Länsi-Suomen alueella, johon Etelä-Pohjanmaa kuuluu, kymmenen prosenttia pk-yrityksistä käyttää tekoälyä säännöllisesti ja 28 prosenttia satunnaisesti \cite{Yrittajat2025}. Etelä-Pohjanmaan valtakunnallista keskiarvoa heikommat talous- ja tuottavuusluvut näyttävät liittyvän siihen, että alueella on poikkeuksellisen suuri osuus pieniä ja mikroyrityksiä \cite{epliitto2022yrittajyys}. Yrittäjien osuus työllisistä on maan korkein, ja suurten yritysten vähyys heijastuu alhaisempana BKT:nä asukasta kohti, matalampana palkkatasona sekä rajallisempana viennin ja TKI-toiminnan volyymina verrattuna muihin maakuntiin \cite{epliitto2022yrittajyys}. Vaikka yleiset suhdannenäkymät ovat Etelä-Pohjanmaan tilannetta käsittelevän kevään 2025 PK-yritysbarometrin mukaan maan heikoimpien joukossa ja investointiodotukset ovat niin ikään erittäin alhaiset \cite{Ohlsbom2025}, tekoäly voi tarjota uusia ja kriittisiä keinoja tuottavuuden parantamiseen ja alueen yritysten kilpailukyvyn vahvistamiseen. On kuitenkin huomattava, että PK-yritysbarometrissa ei esitetä erillisiä alueellisia tietoja tekoälyn käytöstä tai hyödyistä Etelä-Pohjanmaalla \cite{Ohlsbom2025}. Tämä korostaa tarvetta alueelliselle analyysille, johon tämä luku ja raportti pyrkivät vastaamaan.

\section{Toimialakohtaiset mahdollisuudet ja esimerkit}

Kuten aiemmin käsiteltiin, tekoälyn ja erityisesti generatiivisen tekoälyn taloudellinen potentiaali Suomessa on merkittävä ja jakautuu eri toimialoille \cite{Thelle2024}. Vaikka arviolta 75 prosenttia generatiivisen tekoälyn taloudellisesta potentiaalista sijoittuu palvelualoille, valmistava teollisuus ja perussektorit hyötyvät myös merkittävästi. Tekoäly voi lisätä tuottavuutta kaikilla sektoreilla, mikä on kriittistä Suomen taloudelle, jonka tuottavuuskasvu on ollut hidasta vuodesta 2008 \cite{Thelle2024}. Suomalaiset pk-yritykset näkevät tekoälyssä potentiaalia monilla liiketoiminta-alueilla, kuten tuotteiden ja palveluiden kehittämisessä, tuotannossa, laadunvalvonnassa, myynnissä ja markkinoinnissa, asiakaspalvelussa sekä talous- ja IT-toiminnoissa \cite{LahtinenHumala2023}. Tarkastellaan seuraavaksi näitä mahdollisuuksia Etelä-Pohjanmaan keskeisten toimialojen näkökulmasta.

\begin{figure}[h]
\centering
\includegraphics[width=\textwidth]{images/mahdollisuuset.png}
\caption[Toimialakohtaiset mahdollisuudet]{Kolmen toimialan (valmistava teollisuus, elintarvikeala, palvelut/asiantuntijatyö) generatiivisen tekoälyn keskeiset käyttökohteet ja mahdollisuudet.}
\label{fig:toimialat}
\end{figure}

\subsection{Metalliala ja valmistava teollisuus}

Metalliteollisuus ja laajempi valmistava teollisuus ovat elintärkeitä Etelä-Pohjanmaan elinkeinorakenteelle. Suomessa 99 prosenttia kaikista teollisista yrityksistä on pk-yrityksiä, mikä korostaa niiden keskeistä roolia kansantaloudessa \cite{TEM2021}. Tämän sektorin modernisointi on kansallinen prioriteetti. Työ- ja elinkeinoministeriön (TEM) Tekoäly 4.0 -ohjelman yhtenä päätavoitteena onkin kasvattaa digitaalisesti edistyneiden suomalaisten teollisten pk-yritysten joukkoa ja nostaa Suomi tekoälyä ja robotiikkaa soveltavien maiden kärkeen \cite{TEM2021, Tyoe_elinkeinoministerio2022}.

Tämän kansallisen tavoitteen rinnalla myös toimiala itse panostaa voimakkaasti kehitykseen. Esimerkiksi Teknologiateollisuus ry ilmoitti vuonna 2025 merkittävästä kymmenen miljoonan euron investoinnista tekoälysiirtymän vauhdittamiseksi ja perusti tavoitetta tukemaan AI Finland -verkoston. Tavoitteena on nimenomaan tukea tekoälyn kehittämistä ja käyttöönottoa suomalaisissa teknologiayrityksissä ja vahvistaa soveltavaa tutkimusta \cite{Teknologiateollisuus2025a}.

Tekoälyllä on valtava potentiaali teollisuuden prosessien optimoinnissa, laadunvalvonnassa ja ennakoivassa huollossa \cite{LahtinenHumala2023, BusinessFinland2025}. Konkreettisia esimerkkejä on jo nähtävissä: pk-yritykset hyödyntävät tekoälyratkaisuja esimerkiksi uusien autojen maahantulologistiikassa, myynnin ja myyntikatteen kasvattamisen analytiikassa, koneiden huoltotarpeiden ennakoinnissa, muovin valmistusprosessin tehostamisessa konenäön avulla, tilausprosessien automatisoinnissa merkki- ja kuvantunnistuksella, työvuorosuunnittelun automatisoinnissa sekä viemäriverkostojen kuntokartoituksissa konenäön avulla \cite{LahtinenHumala2023}. Finnish AI Region (FAIR) -hankkeen analyyseissa suomalaisista valmistavan teollisuuden yrityksistä on etsitty tekoälyratkaisuja automaattiseen tarjouslaskentaan, koneenosien visuaaliseen tarkastukseen ja objektien automaattiseen tunnistamiseen \cite{Khan2025}. Myös tämän raportin hankkeessa \textit{vAI:lla tuottavuutta?} \cite{seamkvailla} tekoälypohjaisina tuottavuuspilotteina on tehty tilausten virheiden etsintä ja sahanterän vaihtohetken tunnistaminen. Tekoäly nähdään myös keskeisenä välineenä kestävän automaation edistämisessä, jossa keskitytään kustannusten vähentämiseen ja toimintojen tehostamiseen \cite{Chaudhuri2022}.

Valtakunnallisesti teollisuuden pk-yrityksistä yhdeksän prosenttia käyttää tekoälyä säännöllisesti ja 30 prosenttia satunnaisesti. Lisäksi 45 prosenttia suunnittelee lisäävänsä käyttöä seuraavan vuoden aikana \cite{Yrittajat2025}. Tarve on tunnistettu vahvasti myös Etelä-Pohjanmaalla. Alueen yrityksille suunnatussa Osaamisbarometri 2024 -kyselyssä tekoälyosaaminen nousi suurimmaksi yksittäiseksi henkilöstön osaamisen kehittämistarpeeksi (39 \% vastaajista). Teollisuus oli kyselyssä vahvimmin edustettu toimiala (39 \% vastaajista) \cite{SeAMK2025}. Kun digitaalisia taitoja eriteltiin tarkemmin, kärkeen nousivat "tuotantoautomaatio ja robotiikka" (55 \%) sekä "tekoäly ja sen integrointi prosesseihin" (54 \%) \cite{SeAMK2025}. Tämä korostaa sitä, että vaikka järjestelmällinen hyödyntäminen olisi alueella vasta aluillaan, teollisuusyritykset näkevät tekoälyn keskeisenä välineenä erityisesti työn suunnittelun, simulaatio-ohjelmien, työolojen ja työturvallisuuden parantamisessa \cite{Nieminen2023}.

Sekä kansalliset Tekoäly 4.0 - ja Industry 5.0 -ohjelmat \cite{Tyoe_elinkeinoministerio2022} että toimialan omat aloitteet, kuten AI Finland -verkosto \cite{Teknologiateollisuus2025a}, keskittyvät nyt valmistavan teollisuuden pk-yritysten tukemiseen. Näitä kansallisia aloitteita täydentävät alueellisesti paikalliset kehittäjätoimijat, kuten Seinäjoen ammattikorkeakoulu (SeAMK) ja Seinäjoen yliopistokeskus (UCS), jotka tekevät yhteistyötä alueen elinkeinoyhtiöiden kanssa. Näiden tavoitteena on edistää tekoälyn ja muiden digitaalisten teknologioiden kehittämistä ja käyttöönottoa, jotta Suomen teollisuudesta saadaan puhdas, tehokas ja digitaalinen vuoteen 2030 mennessä \cite{Tyoe_elinkeinoministerio2022}.

\subsection{Elintarviketuotanto}

Elintarviketuotanto on Etelä-Pohjanmaan keskeisimpiä teollisuuden aloja, ja sen pk-yritykset ovat kansallisen teollisuuden trendin mukaisesti usein kasvu- tai voimakkaasti kasvuhakuisia ja toimivat kansainvälisen kaupan piirissä \cite{Ohlsbom2025}. Tästä kasvuhakuisuudesta huolimatta koko elintarviketeollisuuden investoinnit ovat kuitenkin viime aikoina kohdistuneet pääasiassa vanhan kapasiteetin korvaamiseen ja sääntelyn vaatimusten täyttämiseen; vain pieni osa (9~\%) investoinneista on suunnattu suoraan kasvuun \cite{Elintarviketeollisuusliitto2025a}.

Tämä haaste korostaa tarvetta tuottavuusloikalle ja prosessien tehostamiselle, joissa tekoäly nähdään strategisena mahdollistajana. Vuonna 2024 julkaistu TKI-tiekartta tunnistaa juuri digitalisaation ja tekoälyn yhdeksi keskeiseksi alueeksi, jossa tarvitaan uusia avauksia ja erityisesti pk-yritysten osaamisen ja resurssien vahvistamista kilpailukyvyn turvaamiseksi \cite{ETL2024}.

Vaikka generatiivisen tekoälyn suurin taloudellinen potentiaali Suomessa kohdistuu palvelualoille, maatalous ja perussektorit voivat saavuttaa jopa miljardin euron lisäarvon tekoälyn avulla esimerkiksi ennakoivassa kunnossapidossa \cite{Thelle2024}. Itse elintarviketeollisuudessa tekoäly voi tehostaa tuotannon suunnittelua, hävikin hallintaa, sekä raaka-aineiden hintojen ennustamista ja tuotteiden hinnoittelun optimointia \cite{BusinessFinland2025}.

Erityisen konkreettinen digitalisaation ja datan hyödyntämisen kohde on ruokaketjun jäljitettävyys, joka on keskeinen kilpailutekijä. Meneillään onkin laaja kansallinen hanke, jossa on mukana myös alueen kannalta merkittäviä toimijoita, kuten Atria ja HK Foods. Hankkeen tavoitteena on luoda Suomeen maailman kattavin, standardoituun dataan perustuva jäljitettävyysjärjestelmä \cite{GS1_Finland2024}. Tekoäly tarjoaa siten eteläpohjalaisille elintarvikealan pk-yrityksille strategisia työkaluja, joilla voidaan vastata alan investointihaasteisiin \cite{Elintarviketeollisuusliitto2025a} ja toteuttaa niiden kasvutavoitteita \cite{Ohlsbom2025}.

\subsection{Palvelut, asiantuntijatyö ja markkinointi}

Palvelualat ja asiantuntijatyö ovat usein tekoälyn hyödyntämisessä edelläkävijöitä, ja niiden merkitys korostuu myös Etelä-Pohjanmaan monimuotoisessa elinkeinorakenteessa. Keväällä 2024 tekoälyteknologioiden käyttö oli Suomessa yleisintä informaation ja viestinnän (66 \%) sekä ammatillisen, tieteellisen ja teknisen toiminnan (48 \%) toimialoilla \cite{Tilastokeskus2024}. Asiantuntijapalveluissa tekoälyn hyödyntäminen on yleisintä pk-yrityksissä: 19 prosenttia käyttää sitä säännöllisesti ja 37 prosenttia satunnaisesti, nähden suurimmat hyödyt toiminnan tehostamisessa (52 \%) ja ajan vapautumisessa (44 \%) \cite{Yrittajat2025}. Myös yksinyrittäjät käyttävät tekoälyä lähes yhtä usein kuin pk-yritykset keskimäärin \cite{Yrittajat2025}.

Generatiivisella tekoälyllä on merkittävä potentiaali asiantuntijatyössä ja palvelualoilla kiihdyttää innovaatioita ja lisätä tuottavuutta automatisoimalla ei-toistuvia tehtäviä \cite{Krapi2024}. Tämä potentiaali on saavutettavissa jopa pienillä investoinneilla \cite{Krapi2024}. Esimerkkejä sovelluksista Etelä-Pohjanmaan pk-yritysten kontekstiin soveltaen ovat:
\begin{itemize}
    \item \textbf{Asiakaspalvelu:} Generatiiviseen tekoälyyn perustuvat työkalut voivat parantaa vastausaikoja ja palvelun laatua. Esimerkiksi matkailu- ja ravintola-alan yritykset voivat vastata brändin mukaisella sävyllä yrityksen omasta tietokannasta \cite{Jantti2025}. Komponentteja valmistava yritys voi tuottaa tarkkoja vastauksia tuotteistaan chatbotin avulla syöttämällä tekniset tiedot tekoälymallin tietokantaan \cite{KauttonenAsikainen2024}.
    \item \textbf{Markkinointi ja myynti:} Suomalaiset markkinointiammattilaiset hyödyntävät generatiivista tekoälyä aivoriihiin, tekstien jalostamiseen, verkkosivujen kehittämiseen (esim. HTML-koodinpätkät, avainsana- ja metatietoehdotukset) ja A/B-testauksen nopeuttamiseen \cite{Ilonen2025}. Esimerkiksi eräs teknisten ja teollisuustekstiilien valmistusyritys käyttää maksullista ChatGPT 4.0 -lisenssiä markkina- ja kilpailija-analyysien tekemiseen kansainvälistymisen tueksi \cite{Jantti2025}.
    \item \textbf{Asiantuntijatyön tehostaminen:} Arkkitehtipalveluita tarjoava yritys voi hyödyntää ohjelmistojen sisäisiä tekoälyominaisuuksia visualisoinnissa ja kuvankäsittelyssä, mikä nopeuttaa työvaiheita ja vapauttaa asiantuntijoiden aikaa ydintehtäviin \cite{Jantti2025}. Samoin rakennuskoneiden ja -laitteiden vuokrausyritys voi automatisoida laskutusprosessinsa tekoälysovelluksen avulla, mikä nopeuttaa myyntisaamisten kiertoa \cite{Jantti2025}. Taloushallinnon alalla generatiivinen tekoäly voi puolestaan auttaa suurten datamäärien käsittelyssä ja automaatioasteen nostamisessa, siirtäen painopistettä asiantuntija- ja asiakaspalvelurooleihin \cite{Kuuva2024}.
    \item \textbf{Logistiikka ja palveluohjaus:} Esimerkiksi hinaus- ja tiepalvelualan pk-yrityksessä tekoälyllä on suuri potentiaali tilauskäsittelyn, tilausten osoittamisen kuljettajille, tarjouslaskennan, sijainnin selvittämisen ja kysynnän ennustamisen tehostamisessa. Tekoäly voisi integroida tilaus- ja karttaohjelmistot ehdottamaan sopivinta kuljettajaa ja automatisoida asiakkaan sijaintitiedon välityksen, vähentäen virheitä ja nopeuttaen avun saamista \cite{Ruusunen2025}. Lisäksi kuljetusten optimointi on klassisen algoritmisen tekoälyn keskeinen sovelluskohde, jossa matemaattisia ja heuristisia menetelmiä hyödynnetään reitityksen ja verkostojen tehokkuuden parantamiseen, ja jota on edelleen kehitetty agenttipohjaisten simulaatioiden ja älykkään päätöksenteon avulla \cite{Kristianto2014, HeloRouzafzoon2023}.
\end{itemize}

\section{Synteesi alueen erityispiirteistä}

Etelä-Pohjanmaan pk-yritysten tekoälymatkaa leimaavat sekä tunnistetut haasteet että merkittävät mahdollisuudet. Alueen yleiset negatiiviset suhdannenäkymät ja investointiodotukset \cite{Ohlsbom2025} korostavat akuuttia tarvetta löytää uusia keinoja tuottavuuden parantamiseen ja kilpailukyvyn vahvistamiseen. Tekoäly, ja erityisesti generatiivinen tekoäly, tarjoaa tähän ratkaisuja mahdollistamalla prosessien tehostamisen ja jopa uusien liiketoimintamallien synnyn pienilläkin investoinneilla \cite{Krapi2024}.

Alueen vahva metalliala ja valmistava teollisuus ovat avainasemassa tekoälyn hyödyntämisessä. Vaikka nämä toimialat ovat historiallisesti olleet hitaampia digitaalisten teknologioiden käyttöönotossa verrattuna palvelusektoriin \cite{TEMAI40_2021}, valtakunnalliset ohjelmat, kuten Tekoäly 4.0, pyrkivät kuromaan tätä eroa umpeen kohdennetuilla tukitoimilla. Ohjelmassa esitetään pk-yrityksille suunnattua tukea tuottavuuden ja kestävyyden parantamiseksi \cite{Jantti2025, TEMAI40_2021}, ja tuen muotoina korostuvat käytännönläheiset palvelut, kuten yritysten tarpeita kartoittava digiagenttitoiminta, kokemusten jakoon perustuvat vertaisryhmät sekä palveluntarjoajien ja yritysten kohtaamista tehostava kehitysfoorumi \cite{Jantti2025}. On kriittistä, että Etelä-Pohjanmaan teolliset yritykset hyödyntävät näitä tukimahdollisuuksia ja ottavat rohkeasti käyttöön tekoälyä esimerkiksi prosessien optimoinnissa, laadunvalvonnassa ja ennakoivassa huollossa \cite{Nieminen2023}. Alueen teollisuuden kasvuorientoituneisuus ja kansainvälinen toiminta, kuten vienti ja osallistuminen EU-tason verkostoihin, luovat otollisen maaperän tekoälyn tukemille innovaatioille ja kilpailukyvyn vahvistamiselle \cite{Ohlsbom2025, TEMAI40_2021}.

Palvelualoilla ja asiantuntijatyössä tekoäly, erityisesti generatiivinen tekoäly, on jo osoittanut hyötyjä markkinoinnissa, asiakaspalvelussa ja sisäisten prosessien tehostamisessa \cite{Jantti2025, Ilonen2025, Yrittajat2025}. Myös yksinyrittäjille, joita Etelä-Pohjanmaalla on runsaasti, tekoäly tarjoaa käytännön työkaluja esimerkiksi sisällöntuotantoon, käännöksiin, asiakaspalveluun ja hallinnollisten tehtävien automatisointiin. Näiden avulla voidaan säästää aikaa, tehostaa toimintaa ja parantaa asiakaspalvelun laatua \cite{Yrittajat2025}.

Alueellisten tukipalveluiden ja ekosysteemien, kuten Euroopan digitaalisten innovaatiohubien (EDIH) ja Finnish AI Region (FAIR) -verkoston, merkitys korostuu. Myös Etelä-Pohjanmaalla on oltu aktiivisia tällä saralla: Into Seinäjoki, SEAMK, Seinäjoen yliopistokeskuksen toimintaa koordinoiva Tampereen yliopisto ja yliopistokeskuksella toimiva Vaasan yliopisto olivat mukana SIX Manufacturing -konsortiossa, joka oli ehdolla EDIH-ohjelmaan. Nämä verkostot tarjoavat matalan kynnyksen apua pk-yrityksille tekoälyn käyttöönotossa ja auttavat yrityksiä keräämään pääomaa tekoälykehitykseen \cite{Asikainen2025, Khan2025}. Näiden ekosysteemien ja vertaisryhmien tuki voi auttaa Etelä-Pohjanmaan yrityksiä kaventamaan osaamisvajetta ja edistämään tekoälyn integroitumista osaksi liiketoimintaa sekä luomaan edellytyksiä kestävälle kasvulle ja innovaatioille. Tässä luvussa tunnistetut alueelliset erityispiirteet ja haasteet luovat perustan luvussa 4 syvennyttäville käyttöönoton esteille ja ratkaisumalleille sekä luvussa 5 esitettäville käytännön suosituksille.