\chapter{Johdanto}

Digitalisaation ja teknologisen kehityksen kiihtyessä tekoäly (AI) on noussut yhdeksi aikamme merkittävimmistä innovaatioista, jolla on potentiaalia muokata syvällisesti yritysten toimintaa ja kilpailukykyä. Erityisesti viime vuosina yleistynyt \textbf{generatiivinen tekoäly (GenAI)} on mullistanut tietotyötä ja avannut uusia, aiemmin ennennäkemättömiä mahdollisuuksia eri toimialoille \cite{Krapi2024, KauttonenAsikainen2024}. Tämä raportti tarkastelee tekoälyn, ja erityisesti generatiivisen tekoälyn, käyttöönottoa, hyötyjä ja haasteita \emph{suomalaisissa pienissä ja keskisuurissa yrityksissä (pk-yrityksissä)}, keskittyen syvällisemmin \emph{Etelä-Pohjanmaan alueeseen} ja sen keskeisiin toimialoihin. Raportti on laadittu osana \textit{vAI:lla tuottavuutta?} -hanketta, joka tukee alueen pk-yrityksiä tekoälyn käyttöönotossa ja hyödyntämisessä. Hanke on Euroopan unionin osarahoittama.

\section{Tarkoitus ja tausta}
Suomen talouskasvun haasteiden ja globaalin kilpailun kiristymisen valossa tekoäly tarjoaa pk-yrityksille kriittisen mahdollisuuden parantaa tuottavuutta, vastata resurssihaasteisiin ja luoda uutta kilpailuetua \cite{Jantti2025}. Suomi on kansainvälisesti korkealle arvostettu tekoälyn kehityksessä. Esimerkiksi Tortoise Median Global AI Index -vertailussa Suomi sijoittui kuudenneksi Euroopassa vuonna 2024 ja Stanfordin Global Vibrancy Ranking -listauksessa viidenneksi väkilukuun suhteutettuna \cite{BusinessFinland2025}. Tekoälyn laajamittainen hyödyntäminen pk-yrityksissä on kuitenkin edelleen alkuvaiheessa. Tilastokeskuksen mukaan 24 prosenttia suomalaisista yrityksistä hyödyntää tekoälyteknologioita \cite{Tilastokeskus2024}, ja PK-yritysbarometrin mukaan 20 prosenttia pk-yrityksistä käyttää tekoälysovelluksia ja robotiikkaa \cite{Ohlsbom2025}. Tuoreen Yrittäjägallupin (2024) mukaan jopa 60 prosenttia suomalaisista pk-yrityksistä ei käytä tekoälyä lainkaan, ja vain 10 prosenttia hyödyntää sitä säännöllisesti \cite{Yrittajat2025}. Käyttö painottuu erityisesti käännöksiin, ideointiin, markkinointiin ja viestintään, mutta myös tuotekehitys, datan analysointi, myynti, asiakasymmärrys ja asiakaspalvelu nousevat esiin \cite{Yrittajat2025}. Tämä osoittaa, että merkittävä osa tekoälyn potentiaalista jää edelleen hyödyntämättä pk-sektorilla, mikä luo kiireellisen tarpeen kohdennetulle tiedolle ja tuelle \cite{Jantti2025}.

Suomen tunnistettuja digitalisaatiohaasteita, kuten verrattain matalia digi-investointeja ja pk-yritysten hidasta arvonluontia, on pyritty ratkomaan useilla kansallisilla ohjelmilla \cite{TEMAI40_2021}. Esimerkiksi Työ- ja elinkeinoministeriön Tekoäly 4.0 -ohjelma (2020--) on keskittynyt vauhdittamaan tekoälyn ja muiden digitaalisten teknologioiden käyttöönottoa, erityisesti valmistavan teollisuuden pk-yrityksissä, tavoitteena tehdä Suomesta edelläkävijä kaksoissiirtymässä (digitalisaatio ja vihreä siirtymä) vuoteen 2030 mennessä \cite{TEM2021, Tyoe_elinkeinoministerio2022}. Myös Business Finland tukee strategisesti generatiivisen tekoälyn käyttöönottoa pk-yrityksissä sekä keskisuurissa, enintään 300 miljoonan euron liikevaihdon yrityksissä kilpailukyvyn vahvistamiseksi \cite{BusinessFinland2024}. Lisäksi erilaiset hankekokonaisuudet, kuten Haaga-Helian AI-TIE (2021–2023) ja Euroopan digitaaliset innovaatiohubit (EDIH), kuten Suomen Finnish AI Region (FAIR), tarjoavat konkreettista tukea ja osaamista pk-yritysten tekoälymatkalle \cite{LahtinenHumala2023, Seppala2023, Asikainen2025}.

OECD:n, BCG:n ja INSEADin tuore tutkimus (2025) \cite{OECD2025} tarkastelee tekoälyn käyttöönottoa G7-maiden ja Brasilian yrityksissä. Sen esiin nostamat teemat, kuten julkisten tukipalveluiden rooli osaajapulan lieventämisessä ja investointien vaikutusten arviointi, ovat erittäin ajankohtaisia myös suomalaisille yrityksille. Suomessa ja erityisesti pk-yrityksissä generatiivisen tekoälyn soveltaminen on kuitenkin vielä uutta \cite{Krapi2024}. Tämä raportti on katsaus tekoälyn mahdollisuuksiin ja haasteisiin pk-yritysten näkökulmasta. Raportti keskittyy Etelä-Pohjanmaan alueellisiin ja toimialakohtaisiin erityispiirteisiin, perustuu ajantasaiseen kirjallisuuskatsaukseen ja tarjoaa käytännön suosituksia yrityksille ja päättäjille.

\begin{figure}[h]
\centering
\includegraphics[width=\textwidth]{images/rakenne.png}
\caption[Raportin rakenne]{Raportin kokonaisrakenne ja etenemispolku: tausta ja tavoitteet, katsaus tutkimustietoon, alueellinen nykytila, haasteet ja ratkaisut sekä suositukset.}
\label{fig:rakenne}
\end{figure}

\section{Tavoitteet ja tutkimuskysymykset}
Tämän raportin ensisijaisena tavoitteena on tarkastella tekoälyn (erityisesti generatiivisen tekoälyn) käyttöönoton nykytilaa, mahdollisuuksia ja haasteita suomalaisissa pk-yrityksissä, erityisesti Etelä-Pohjanmaan näkökulmasta. Raportti tunnistaa keskeisiä havaintoja ja tarjoaa käytännön suosituksia pk-yrityksille sekä alueellisille päättäjille ja sidosryhmille.

Raportti pyrkii vastaamaan seuraaviin keskeisiin tutkimuskysymyksiin:
\begin{enumerate}
    \item Mitkä ovat tekoälyn (erityisesti generatiivisen tekoälyn) käyttöönoton vaikutukset ja hyödyt suomalaisissa pk-yrityksissä?
    \item Mitkä ovat kirjallisuuden keskeiset löydökset, haasteet ja mahdollisuudet?
    \item Miten tekoälyn käyttöönotto eroaa pk-yrityksissä verrattuna suurempiin yrityksiin ja eri toimialoilla (esim. teollisuus, palvelut, toimistotyö, markkinointi, elintarviketuotanto, metalliala)?
    \item Millaisia käytännön esimerkkejä, tapaustutkimuksia tai empiiristä näyttöä Suomesta löytyy?
    \item Mitkä ovat Etelä-Pohjanmaan ja sen keskeisten toimialojen (esim. yksityisyrittäjät, metalliteollisuus, elintarviketuotanto) erityispiirteet tekoälyn käyttöönottoa ajatellen ja millaisiin johtopäätöksiin ne ohjaavat?
\end{enumerate}

\subsection{Menetelmä ja rajaus}
Tämä raportti on toteutettu kattavana kirjallisuuskatsauksena, jonka tavoitteena on syntetisoida olemassa olevaa tutkimustietoa ja ajankohtaisia selvityksiä tekoälyn käyttöönotosta pk-yrityksissä. Tiedonkeruu on painottunut vertaisarvioituihin tieteellisiin artikkeleihin, konferenssijulkaisuihin, systemaattisiin katsauksiin sekä arvostettuihin tutkimusraportteihin ja selvityksiin (esim. OECD, Business Finland, EK, Haaga-Helia) \cite{Jantti2025, BusinessFinland2025, LahtinenHumala2023, OECD2025}. Lisäksi on hyödynnetty suomalaisia soveltavan tutkimuksen julkaisuja ja opinnäytteitä (esim. Theseus-tietokannasta). Aikajänne on keskittynyt ensisijaisesti vuosien 2020–2025 julkaisuihin, mutta relevantteja julkaisuja vuodesta 2018 alkaen on myös huomioitu.

Raportin pääasiallinen kohderyhmä ovat pienet ja keskisuuret yritykset (pk-yritykset), jotka määritellään alle 250 työntekijän ja alle 50 miljoonan euron liikevaihdon yrityksiksi. Vaikka painopiste on pk-yrityksissä, vertailutietoina hyödynnetään myös suuria yrityksiä koskevia tutkimuksia \cite{OECD2025}. Metodologisesti on tarkasteltu sekä laadullisia (esim. tapaustutkimukset, haastattelut) että määrällisiä (esim. kyselyt, tilastollinen analyysi) tutkimuksia.

Toimialarajaus kohdistuu Etelä-Pohjanmaalle tärkeisiin aloihin, kuten metalli- ja valmistavaan teollisuuteen sekä merkittävään elintarviketuotantoon. Näiden lisäksi raportissa tarkastellaan palveluiden, toimistotyön ja markkinoinnin aloja. Tämä rajaus mahdollistaa syvällisen ja käytännönläheisen tarkastelun tekoälyn mahdollisuuksista ja haasteista raportin kohderyhmän näkökulmasta.

Raportin rakenne on seuraava: \textbf{Luku 2} määrittelee tekoälyn ja generatiivisen tekoälyn käsitteet sekä esittelee niiden yleiset hyödyt ja sovelluskohteet. \textbf{Luku 3} siirtyy yleiseltä tasolta alueelliseen ja toimialakohtaiseen analyysiin, keskittyen Etelä-Pohjanmaan nykytilaan ja potentiaaliin. \textbf{Luku 4} erittelee käyttöönoton keskeiset haasteet ja esittelee niihin ratkaisumalleja. Lopuksi \textbf{luku 5} kokoaa yhteen raportin johtopäätökset ja esittää konkreettiset suositukset sekä eteläpohjalaisille pk-yrityksille että alueen kehittäjille.