\chapter{Conclusion, Future Work, and Takeaways}
This chapter summarizes the accomplishments of the project, highlighting the successful integration of multiple AI models into a functional voice-guided application. It discusses the outcomes and lessons learned from the implementation and outlines areas for future work, including scalability, advanced features, and user experience improvements. The chapter concludes with key takeaways that emphasize the potential and challenges of locally hosted generative AI systems.

\section{Summary of Accomplishments}  
This project successfully brought together several open-source AI models to create a functional application capable of processing voice commands to generate and edit images and videos. Key accomplishments include real-time transcription, translation, and intent recognition using Faster-Whisper and Mistral, which facilitated the voice-driven interactions. Responsiveness was maintained through partial updates and feedback mechanisms, even during long-running tasks like image editing and video generation. Memory-efficient deployment strategies, such as quantization and dynamic model loading, enabled the integration of multiple models on consumer-grade hardware. Furthermore, demonstration scenarios showcased the practical utility of the application for voice-guided multimedia generation and editing, emphasizing its potential for real-world use cases.

\section{Outcomes}  
The project demonstrated the feasibility of locally hosting and integrating generative AI models while balancing performance and memory constraints. Optimization techniques, including quantization, CPU offloading, and slicing, were applied to enhance resource efficiency without compromising the quality of results. Additionally, the implementation highlighted the critical role of user instructions and interface design in improving system usability and minimizing errors. These findings provide a robust foundation for future efforts aimed at scaling and refining AI systems to address broader applications and use cases.

\section{Future Work and Extensions}  
Although the project achieved its primary goals, several areas remain open for further exploration and improvement. Scalability is a critical avenue for future work, including the development of multi-user support through session management, distributed processing, or cloud-based deployments to handle concurrent workloads. Enhancing image and video editing functionalities, such as introducing region-specific edits and complex style transformations, would further expand the application’s capabilities. Improved error handling mechanisms are also necessary to interpret vague commands, offer suggestions, and dynamically clarify user intent.

Future research should also conduct detailed comparisons between locally hosted and cloud-hosted AI models, focusing on cost analysis, latency benchmarks, and overall performance. Enhancements to user experience, including more accessible UI design and streamlined workflows, would help make the application more intuitive for non-technical users. Lastly, this project could serve as a foundation for a master’s thesis, providing an opportunity to delve deeper into AI deployment strategies, with a focus on scalability, usability, and cost-efficiency.

\section{Key Takeaways}  
This project underscores the potential of integrating multiple generative AI models into a practical, voice-driven application. It demonstrated that locally hosting AI models is feasible, though it requires careful optimization of memory usage and performance to ensure reliability on consumer hardware. Combining speech recognition, intent parsing, and generative models opens new possibilities for multimodal applications, but these integrations demand thoughtful design and execution.

Critical optimization techniques, such as quantization and dynamic loading, proved essential for deploying large models on limited hardware. However, transitioning this proof-of-concept into a production-ready system will require advancements in scalability, editing features, and user interaction design. Despite these challenges, the project serves as a promising starting point for further research and development in AI-driven multimedia tools, providing insights into both the technical obstacles and the practical solutions needed for future innovations.

\section{ITC.CEE.300 Special Topics}
This section is exclusively included for the ITC.CEE.300 Special Topics Final Report submission. The project was done at the request of and carried out under the guidance of Dr. Jussi Rasku, as part of the GPT Lab Seinäjoki AI research initiative. The original study plan for this project was approved by Timo Poranen.

The objective of this project was to explore the use of locally hosted generative AI models for voice-guided multimedia applications. The primary deliverable was a working demo that integrated multiple AI models, such as Whisper for speech transcription, Mistral for intent recognition, and Stable Diffusion-based models for image and video generation. The project required extensive research into the technical requirements, capabilities, and performance of locally hosted generative AI systems.

Overall, it took around 36 hours of studying and research, 108 hours for implementing the application and 42 hours for writing this report, bringing the total work hours for this project to around 186 hours.

This project provided significant learning opportunities, especially in the practical application of generative AI technologies. Key insights gained include understanding how generative AI models operate, the requirements for hosting such models locally, and the optimization techniques required to ensure their performance. Additionally, the work involved extensive learning about Python libraries, open-source communities, and platforms for running AI models.

The report itself was written using Overleaf and LaTeX, tools that were new to the author. This process enhanced technical writing skills and provided invaluable experience for future academic endeavors, particularly in preparing a master’s thesis on this topic.

The work aligns with the goals outlined in the original study plan, combining practical learning with innovative application development. The hands-on experience gained from integrating and optimizing generative AI models has laid a strong foundation for future research and development in AI-driven applications.