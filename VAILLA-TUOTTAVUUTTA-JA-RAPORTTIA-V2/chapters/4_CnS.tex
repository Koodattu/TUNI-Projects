\chapter{Käyttöönoton haasteet ja ratkaisumallit}

Vaikka tekoäly tarjoaa merkittäviä mahdollisuuksia, kuten luvuissa 2 ja 3 käsiteltiin, sen laajempaan hyödyntämiseen liittyy edelleen useita tunnistettuja haasteita. Nämä esteet voivat syventää kilpailukyvyn kuilua yritysten välillä. Tässä luvussa käsitellään näitä haasteita sekä esitellään niihin konkreettisia ratkaisumalleja, jotka muodostavat pohjan luvun 5 käytännön suosituksille.

\section{Tunnistetut esteet}

Pk-yritykset kohtaavat \textbf{tekoälyn} käyttöönotossa usein laaja-alaisia haasteita, jotka poikkeavat osittain suuryritysten kohtaamista esteistä rajallisempien resurssiensa vuoksi \cite{TEM2021, Tyoe_elinkeinoministerio2022}. Vaikka kiinnostus \textbf{tekoälyä} kohtaan kasvaa jatkuvasti, monilla pk-yrityksillä on edelleen merkittäviä esteitä teknologian täysimääräiselle hyödyntämiselle.

\begin{figure}[h]
\centering
\includegraphics[width=\textwidth]{images/ktnotto.png}
\caption[Esteet käyttöönottolle]{Generatiivisen tekoälyn käyttöönoton esteet kalanruotokaaviona: osaaminen, kustannukset, data, muutosvastarinta, tietoturva ja etiikka, sääntely, tuottojen ja investointien suhteen (ROI) epävarmuus sekä integraatio.}
\label{fig:esteet}
\end{figure}

\subsection{Osaamisen ja tiedon puute}

Osaamisen ja tiedon puute on ylivoimaisesti suurin yksittäinen este tekoälyn käyttöönotolle suomalaisissa pk-yrityksissä. Elinkeinoelämän Keskusliiton (EK) kyselyssä jopa 68 prosenttia yrityksistä mainitsee tämän esteen \cite{Jantti2025}. Tämän vahvistavat myös muut selvitykset: PK-yritysbarometrin mukaan 48 prosenttia ja Yrittäjägallupin mukaan 44 prosenttia pk-yrityksistä koki tiedon tai osaamisen puutteen hidastavan käyttöönottoa \cite{Ohlsbom2025, Yrittajat2025}. Ongelma korostuu yrityskoon kasvaessa; yli kymmenen hengen pk-yrityksistä jo 60 prosenttia nimeää osaamisen puutteen esteeksi \cite{Yrittajat2025}. Alueellisella tasolla, esimerkiksi Etelä-Pohjanmaalla, tekoälyosaaminen nousi selkeästi suurimmaksi henkilöstön kehittämistarpeeksi (39 \%) \cite{SeAMK2025}.

Ilmiö ei ole vain alueellinen. Kansainväliset selvitykset osoittavat, että noin 85 prosenttia EU-alueen yrityksistä ilmoittaa vaikeuksista löytää ja palkata tekoälytehtäviin pätevää henkilöstöä, ja jopa 20 prosenttia 50–250 työntekijän yrityksistä ei löytänyt lainkaan sopivia hakijoita \cite{OECD2025}. Koulutustarjonta ei ole pysynyt kysynnän perässä: vain 19 prosenttia suomalaisista työntekijöistä on itseopiskellut tai saanut työnantajan tarjoamaa tekoälykoulutusta, mikä on vähemmän kuin muissa Pohjoismaissa \cite{Thelle2024}. Generatiivisen tekoälyn osalta koulutus on vielä vähäisempää ja perustuu usein omaehtoiseen opiskeluun \cite{Heikkila2025}. Tämä osaamisvaje heijastuu suoraan yritysten kyvykkyyteen, sillä alhainen tekninen osaaminen on vahvasti yhteydessä yritysten matalaan tekoälyn kypsyystasoon \cite{Khan2025}.

\subsection{Kustannukset ja resurssit}

Pk-yritysten rajalliset resurssit ovat keskeinen hidaste uusien teknologioiden omaksumiselle \cite{Tyoe_elinkeinoministerio2022, LahtinenHumala2023, Lahtinen2024}. Tuoreimman Pk-yritysbarometrin mukaan 18 prosenttia pk-yrityksistä kokee tekoälyn käyttöönottoon liittyvät kustannukset esteeksi \cite{Ohlsbom2025}. Yleisemmin digitalisaation esteiksi mainitaan ylläpitokustannukset (43 \%), koulutukseen käytettävissä olevan ajan puute (38 \%) ja laitteistokustannukset (32 \%) \cite{BianchiniSancho2025}.

Korkeat alkuinvestoinnit ohjelmistoihin ja henkilöstön koulutukseen voivat hidastaa käyttöönottoa merkittävästi \cite{Kuuva2024, LahtinenHumala2023}. Esimerkiksi ulkopuolisen toimittajan kehittämän tekoälyprojektin hinta voi nousta jopa \mbox{300000–400000} euroon, mikä on monille pienille yrityksille aivan liian suuri investointi \cite{Alnajjar2025}. Myös kehittyneempien, lisenssimaksullisten \textbf{tekoälytyökalujen} on arvioitu olevan liian kalliita pk-yritysten resursseille, mikä saattaa syventää eriarvoisuutta suurempiin organisaatioihin verrattuna \cite{Ilonen2025}.

\subsection{Datan saatavuus ja laatu}

Datan laatu ja saatavuus ovat kriittisiä tekoälyn onnistuneelle käyttöönotolle. Huonolaatuinen tai siiloutunut data voi estää ratkaisujen toteuttamisen tai johtaa epäluotettaviin tuloksiin \cite{Jantti2025, LahtinenHumala2023, Seppala2023}. Erityisesti pk-yrityksillä on usein haasteena kerätä riittävästi laadukasta ja jäsenneltyä dataa koneoppimismallien kouluttamista varten, mikä hidastaa käyttöönottoa \cite{OECD2025, OECD2024, Krapi2024}. Tehokkaiden tekoälyratkaisujen kehittäminen ja ylläpito edellyttävätkin jatkuvaa ja systemaattista datahallintaa \cite{LahtinenHumala2023, OECD2024}.

\subsection{Muutosvastarinta ja organisaation valmius}

Tekoälyn käyttöönotto ei ole pelkkä teknologinen investointi, vaan kokonaisvaltainen toimintatapojen muutos, joka vaatii vahvaa johtajuutta \cite{Jantti2025}. Organisaation valmiutta heikentää kuitenkin usein merkittävä osaamisen puute. Esimerkiksi Saksassa lähes puolet (48 \%) työnantajista kokee osaamisen puutteen esteeksi tekoälyn käyttöönotolle \cite{OECD2024}. Suomessa luku on vielä korkeampi: 68 \% yrityksistä nimeää tarpeellisen osaamisen puutteen keskeiseksi syyksi olla hyödyntämättä tekoälyä \cite{Jantti2025}. Osaamisvaje koskee näin ollen myös yrityksen johtoa. Ilman johdon sitoutumista, henkilöstön osaamisen kehittämistä ja sitä kautta syntyvää ymmärrystä tekoälyn mahdollisuuksista muutosvastarinta voi hidastaa uusien työkalujen omaksumista \cite{Jantti2025, Seppala2023}. Tutkitusti organisaatioiden ja yksilöiden vastarinta tekoälyn käyttöönotossa syntyy pääasiassa epäluottamuksesta, osaamisvajeesta ja työn sisällön epäselvistä muutoksista. Lisäksi korkeiden päätösvastuiden yhteydessä algoritmeihin kohdistuva epäluottamus kasvaa voimakkaasti \cite{Laumer2024, Mahmud2022}.

\subsection{Tietoturva, etiikka ja sääntely}

Tietoturvaan ja yksityisyyteen liittyvät huolet ovat merkittäviä esteitä tekoälyn, erityisesti generatiivisen tekoälyn, käyttöönotolle. Noin 23 prosenttia pk-yrityksistä pitää tietoturvan vaarantumista esteenä \cite{Ohlsbom2025}. Generatiivisten tekoälybottien käyttöön liittyviä yleisimpiä haasteita ovat tietoturva ja luottamuksellisuus, erityisesti luotettavuuden epäilyt tiedon syöttämisessä \cite{Heikkila2025}. Peräti 83 prosenttia generatiivista tekoälyä käyttämättömistä pk-yrityksistä on huolissaan tiedon yksityisyydestä \cite{BianchiniSancho2025}. Markkinoinnin ammattilaiset ovat varovaisia syöttäessään arkaluontoista tietoa tekoälytyökaluihin tietosuojasyistä \cite{Ilonen2025}. Vuoden 2025 OECD-kyselyn mukaan 32 prosenttia pk-yrityksistä oli kokenut tietoturvaloukkauksen edeltävän vuoden aikana, mikä oli kaksinkertainen määrä vuoden 2024 kyselyyn verrattuna \cite{BianchiniSancho2025}. Myös eettiset huolet (29~\%) ja tiedon virheellisyys ('hallusinaatiot') (79~\%) ovat merkittäviä esteitä \cite{Yrittajat2025, BianchiniSancho2025, Ilonen2025, KauttonenAsikainen2024}.

Sääntelyepävarmuudet, kuten EU:n tekoälyasetus ja GDPR, voivat lisätä investointeihin liittyvää epävarmuutta ja lisätä käyttöönoton kustannuksia \cite{OECD2025, OECD2024}. Esimerkiksi terveydenhuollon tekoälyn osalta EU:n sääntelyä pidetään liian tiukkana, mikä hidastaa innovaatioita \cite{Alnajjar2025}. Onnistuneen käyttöönoton edellytyksenä on kriittinen arviointikyky ja tietoisuus mahdollisista virheistä \cite{Rezazadeh2025}.

\subsection{Sijoitetun pääoman tuoton arvioinnin haasteet}

Yksi yleisimmistä tekoälyn käyttöönoton esteistä on investoinnin tuottavuuden ennakoinnin vaikeus \cite{OECD2025}. Merkittävä osa pk-yrityksistä (46 \%) ei koe saavansa tekoälyn käytöstä lainkaan liiketoiminnallista hyötyä \cite{Ohlsbom2025}. Tämä liittyy usein siihen, että teknologiaa otetaan käyttöön ilman selkeää strategiaa, liiketoiminta-arvoa tai suunnitelmaa laajempaan hyödyntämiseen (”\textbf{tekoälyä tekoälyn vuoksi}”) \cite{Khan2025}.

\subsection{Integraatio- ja datahaasteet}

Teknisenä haasteena korostuvat tekoälyjärjestelmien yhteensopivuusongelmat ja datan siiloutuminen, jotka estävät helpon integroinnin yrityksen olemassa oleviin järjestelmiin \cite{Jantti2025, LahtinenHumala2023, Krapi2024}. Erityisesti valmiiden tekoälyratkaisujen puutteelliset integraatiomahdollisuudet voivat muodostua merkittäväksi hidasteeksi pk-yrityksille \cite{Ruusunen2025}. Lisäksi pelkkä tukeutuminen kolmannen osapuolen geneerisiin tekoälymalleihin ilman omaa, lisäarvoa tuottavaa sovellusta ei välttämättä luo kestävää kilpailuetua \cite{Khan2025}. Toisaalta generatiivinen tekoäly saattaa helpottaa dataintegraatiota tulkitsemalla eri järjestelmien tietoja ja yhdistämällä ne yhteiseen muotoon ilman raskasta käsityötä. Se osaa tulkita ei-rakenteisen datan merkityksiä ja osaa luoda niistä yhtenäisen kokonaisuuden, jota voi käyttää päätöksenteossa ja analytiikassa \cite{Dong2024}.

\section{Matalan kynnyksen ratkaisut ja parhaat käytännöt}

Edellä esitettyjen haasteiden, kuten osaamisen puutteen, taloudellisten rajoitteiden ja muutosvastarinnan, voittamiseksi on tunnistettu useita matalan kynnyksen ratkaisumalleja ja parhaita käytäntöjä. Nämä keskittyvät osaamisen kehittämiseen, kulttuurimuutokseen, yhteistyöhön ja oikeanlaisen julkisen ja ekosysteemituin varmistamiseen, pyrkien madaltamaan pk-yritysten kynnystä tekoälyn laajempaan hyödyntämiseen.

\subsection{Osaamisen kehittäminen ja koulutus}

Tekoälyn käyttöönotossa on tärkeää aloittaa henkilöstön tekoälyvalmiuksien rakentamisesta \cite{Jantti2025}. Suomen Yrittäjien kyselyjen mukaan pk-yrityksiä motivoisi eniten koulutukset (31 \%) ja selkeämmät liiketoimintahyödyt (28 \%) \cite{Yrittajat2025}. Yleisimmin toivottu koulutuksen aihealue on tekoälyn käyttäminen oman työn tehostamisessa (33 \%), toiseksi eniten toivotaan tekoälyn perusteiden koulutusta (30 \%) \cite{Yrittajat2025}. Työnantajan kannustus ja koulutus voivatkin toimia suurimpina kannustimina tekoälysovellusten käyttöönotolle \cite{Kuuva2024}. Osaamisvajeen ratkaisemiseksi on käynnistetty erilaisia oppimiskokonaisuuksia, kuten Suomen Yrittäjien ja Elisan ”Yrittäjät \& Tekoäly” -ohjelma \cite{Yrittajat2025}. Myös jatkuvan oppimisen mallin mukaiset mikrotutkinnot ja monialaiset oppimateriaalit, jotka kattavat tekoälyn ja digitalisaation, ovat suositeltavia \cite{Tyoe_elinkeinoministerio2022}. Johtotehtävissä toimivien tekoälyosaamisen vahvistaminen on erityisen tärkeää \cite{Hajikhani2023}.

\subsection{Yrittäjämäinen kehittämiskulttuuri ja johtajuus}

Pieni yrityskoko voi olla kilpailuetu suuryrityksiin verrattuna tekoälyn käyttöönotossa, sillä pk-yritysten joustavampi rakenne ja toimintatavat mahdollistavat nopeamman päätöksenteon ja muutoksen johtamisen \cite{Jantti2025}. Tekoälyn täysimääräinen hyödyntäminen edellyttää kuitenkin myös organisaation ja toimintatapojen kehittämistä sekä yrityskulttuurin muutosta ja johtajuutta, joka luo dynaamisen, innovatiivisen ja yrittäjämäisen kehittämiskulttuurin \cite{Lahtinen2024, LahtinenHumala2023}. Johtajuudella on ratkaiseva rooli positiivisen ja kannustavan työympäristön luomisessa, resurssien tarjoamisessa innovatiivisille projekteille ja jatkuvan oppimisen tukemisessa \cite{LahtinenHumala2023}. Keskeistä on, että yritysjohtajat osoittavat työntekijöilleen, että tekoäly pikemminkin tehostaa ja täydentää heidän työtään kuin korvaa sitä, mikä vähentää muutosvastarintaa ja edistää hyväksyntää \cite{Rezazadeh2025, Kang2024}. Työntekijöiden proaktiivisuus, työn tuunaaminen ja minäpystyvyys (superkompetenssit) ovat avainasemassa tekoälyn tehokkaassa käyttöönotossa, sillä jokaisen henkilön rooli on merkityksellinen pk-yrityksissä \cite{LahtinenHumala2023}.

\subsection{Yhteistyö ja ekosysteemit}

Verkostoituminen ja yhteistyö sidosryhmien, kuten muiden yritysten, palveluntarjoajien ja asiantuntijaorganisaatioiden, kanssa tarjoaa tukea tekoälyn käyttöönottoon \cite{Jantti2025, LahtinenHumala2023}. Aktiiviseen tekoälyekosysteemiin osallistuvat yritykset osoittavat korkeampaa tyytyväisyyttä tekoälyyn \cite{Jantti2025}. Erilaiset kansalliset ja eurooppalaiset tukiverkostot voivat vastata pk-yritysten todelliseen ja kasvavaan tarpeeseen saada matalan kynnyksen ja korkean asiantuntemuksen tekoälypalveluita \cite{Asikainen2025, Alnajjar2025, Khan2025}. Myös ’digiagenttien’ toiminta ja digikehittämisen vertaisryhmätoiminta teollisille pk-yrityksille on ehdotettu, jotta ne voivat jakaa kokemuksia ja osaamisia \cite{Tyoe_elinkeinoministerio2022}. Alueellisesti Etelä-Pohjanmaalla Seinäjoen yliopistokeskus on kehittänyt tekoälyosaamista hankkeilla, kuten AI-BASS, GeTeK ja GPT-Lab Seinäjoki, jotka tukevat pk-yritysten generatiivisen tekoälyn käyttöönottoa ja osaamisen vahvistamista. Seinäjoen ammattikorkeakoulu puolestaan on edistänyt teollisten pk-yritysten ja koulutuksen tekoälyvalmiuksia hankkeilla, kuten Tekoäly-AKKE, BAIT-HEI, Digital Factory ja tämän raportin hankkeessa \textit{vAIlla tuottavuutta?}.

\subsection{Riskienhallinta ja käytäntöjen dokumentointi}

Vaikka pk-yritykset tunnistavat monipuolisesti tekoälyyn liittyviä riskejä, kuten tietoturvan ja yksityisyydensuojan, ne panostavat yhä vähän riskien aktiiviseen lieventämiseen \cite{LahtinenHumala2023}. Käyttöönotossa on tärkeää ottaa huomioon tietoturvariskit, ja arkaluontoisten tietojen käsittelyyn suositellaan paikallisia tekoälyjärjestelmiä pilvipohjaisten sijaan \cite{Krapi2024, Kuuva2024}. Tehokas syötesuunnittelu ja luodun sisällön kriittinen tarkastelu ovat GenAI:n käytössä kriittisiä taitoja, jotka auttavat välttämään hallusinaatioita ja virheellisen tiedon leviämistä \cite{KauttonenAsikainen2024, Ilonen2025, Rezazadeh2025}. Vuonna 2024 vain 11 prosenttia suomalaisista yrityksistä oli dokumentoinut tekoälyn käyttöön liittyviä ohjeita tai toimintatapoja, mikä korostaa selkeiden käytäntöjen luomisen tarvetta \cite{Tilastokeskus2024}.

\subsection{Rahoitus ja julkinen tuki}

Julkiset rahoitusohjelmat ja tukimekanismit voivat madaltaa tekoälyn käyttöönoton kynnystä pk-yrityksissä. Business Finlandin rahoitushaku generatiiviseen tekoälyyn pk- ja midcap-yrityksille tukee varhaisen vaiheen soveltamiskokeiluja ja uusien ratkaisujen pohjatiedon luomista \cite{BusinessFinland2024}. Myös TKI-verovähennyksen laajentaminen kattamaan kokeiluympäristöiltä hankittavat palvelut sekä eurooppalaisten digitaalisten innovaatiohubien (EDIH) resursointi ovat ehdotettu madaltamaan kokeilukynnystä \cite{TEM2021}. EU:n tekoälysäädös pyrkii keventämään velvoitteita pienen riskin tai minimaalisen riskin tekoälyjärjestelmien osalta, mikä kattaa suurimman osan pk-yritysten käyttötilanteista \cite{Jantti2025, Seppala2023}. Pk-yritysten kilpailukyvyn tukemiseksi on tärkeää tarjota niille tarvittavat työkalut ja kiihdytysohjelmat, jotta voidaan kuroa umpeen kuilu pk-yritysten ja suurempien yritysten välillä tekoälyn hyödyntämisessä \cite{HaagaHelia2023}. Oikeanlaista tukea tarvitaan tekoälyn käyttöönoton eri vaiheissa – ideointiin, osaamiseen, rahoitukseen ja asiantuntijoihin liittyen \cite{Seppala2023}.