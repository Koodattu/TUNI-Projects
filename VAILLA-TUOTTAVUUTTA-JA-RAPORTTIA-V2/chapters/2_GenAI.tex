\chapter{Tekoäly pk-yrityksen mahdollisuutena: Katsaus tutkimustietoon}

Kuten \textbf{luvussa 1} (Johdanto) esitettiin, tekoäly (AI) on noussut digitaalisen murroksen keskiöön, muokaten syvällisesti yritysten toimintaa ja kilpailukykyä. Tämä luku syventyy tarkemmin tekoälyn, ja erityisesti generatiivisen tekoälyn (GenAI), käsitteisiin. Se valottaa niiden keskeisiä hyötyjä ja monipuolisia sovelluskohteita yritysmaailmassa, luoden käsitteellisen perustan tuleville analyyseille. Lisäksi tarkastelemme tekoälyn käyttöönoton yleistä tilaa Suomessa, erityisesti pienissä ja keskisuurissa yrityksissä (pk-yritykset), pohjustaen \textbf{luvun 3} alueellista ja toimialakohtaista tarkastelua.

\section{Mitä on tekoäly ja generatiivinen tekoäly?}

Tekoäly (\textbf{AI}) viittaa ohjelmistoihin ja järjestelmiin, jotka kykenevät suorittamaan perinteisesti ihmisen älyyn liitettyjä tehtäviä, kuten päättelyä, oppimista, suunnittelua ja luovaa työtä \cite{Jantti2025, LahtinenHumala2023}. Ytimeltään se yhdistää edistyneen koneoppimiseen perustuvan analytiikan automaatioon \cite{LahtinenHumala2023}. Tekoälyjärjestelmät hyödyntävät monipuolisesti dataa tuottaakseen ennusteita, suosituksia tai päätöksiä, käyttäen apunaan esimerkiksi tekstinlouhintaa, konenäköä, puheentunnistusta tai koneoppimista \cite{viitaila2025tekoalyn, Jantti2025}. Ne kykenevät myös itsenäisesti oppimaan ja parantamaan toimintaansa algoritmien avulla \cite{viitaila2025tekoalyn, Jantti2025}.

\textbf{Generatiivinen tekoäly} (\textbf{GenAI}) edustaa tekoälyn kehittyneempää ja tällä hetkellä ajankohtaisinta muotoa, jonka laaja leviäminen alkoi vuonna 2022 muun muassa ChatGPT:n myötä \cite{Thelle2024, Ruusunen2025}. Toisin kuin perinteinen, ennusteita ja yksittäisiä päätöksiä tekevä tekoäly, GenAI kykenee luomaan uutta sisältöä, kuten tekstiä, kuvia, videoita, ääntä tai ohjelmistokoodia, käyttäjän antamien ohjeiden eli kehotteiden (prompts) perusteella \cite{Jantti2025, KauttonenAsikainen2024}. Tämä tapahtuu analysoimalla ja syntetisoimalla tietoa valtavista aineistoista, tulkitsemalla ja hyödyntämällä myös ei-rakenteista dataa, sekä vuorovaikuttamalla käyttäjien kanssa lähes reaaliaikaisesti \cite{Kang2024, Thelle2024}. Generatiivisen tekoälyn helppokäyttöisyys on tehnyt edistyksellisestä teknologiasta saavutettavaa myös pk-yrityksille. Tämä erottaa sen aiemmista innovaatioista, jotka olivat usein vain suurten yritysten ulottuvilla \cite{Heikkila2025, Seppala2023}. Helppokäyttöisyyden vuoksi GenAI:n arvioidaan kiihdyttävän automaatiota merkittävästi aiheuttaen tuottavuusloikan, sillä se vaatii aiempiin tekoälymenetelmiin verrattuna pienempiä data- ja infrastruktuuri-investointeja \cite{Thelle2024}.

\begin{figure}[h]
\centering
\includegraphics[width=\textwidth]{images/selite.png}
\caption[Tekoälyn käsitteellinen kartta]{Tekoälyn jaottelu perinteiseen tekoälyyn ja generatiiviseen tekoälyyn esimerkkien kera.}
\label{fig:kasitteet}
\end{figure}

\section{Keskeiset hyödyt ja sovelluskohteet}

Tekoälyn käyttöönotto yrityksissä on ollut viime vuosina voimakkaassa kasvussa sekä kansainvälisesti että Suomessa, ja sen hyödyntäminen nähdään yhä kriittisempänä tekijänä kilpailukyvyn ylläpitämisessä \cite{Lahtinen2024, Jantti2025}. Suomi on noussut Euroopan edelläkävijäksi tekoälyn adoptoinnissa: Eurostatin vuoden 2023 tutkimuksen mukaan Suomessa ja Tanskassa noin 15 prosenttia yrityksistä hyödyntää vähintään yhtä tekoälyteknologiaa, mikä on merkittävästi korkeampi kuin EU-keskiarvo (8 prosenttia) \cite{OECD2025}. Tilastokeskuksen mukaan keväällä 2024 jo 24 prosenttia suomalaisista yrityksistä käytti tekoälyteknologioita, mikä osoittaa nopeaa kasvua vuoden 2020 12 prosentista \cite{Tilastokeskus2024, TEMAI40_2021}. Tämä kehitys heijastaa osin generatiivisen tekoälyn nopeaa yleistymistä ja sen tuomia uusia mahdollisuuksia \cite{Ohlsbom2025, Tilastokeskus2024}.

Pk-yritysten osalta tekoälyn käyttöönotto on niin ikään kiihtynyt: kevään 2025 PK-yritysbarometrin mukaan 20 prosenttia suomalaisista pk-yrityksistä hyödyntää tekoälysovelluksia tai robotiikkaa, ja yhteensä 42 prosenttia on vähintään kokeillut tekoälyä \cite{Ohlsbom2025}. Pienemmissä yrityksissä (alle 51 työntekijää) tekoälyn adoptio on edelleen vähäisempää ja monet ovat vasta tekoälykypsyytensä alkuvaiheessa \cite{KauttonenAsikainen2024, Alnajjar2025, Khan2025}. Merkittäväksi esteeksi laajemmalle käyttöönotolle nousee osaamisen puute, jonka mainitsee 48 prosenttia pk-yrityksistä \cite{Ohlsbom2025}. Tukea käyttöönottoon on kuitenkin saatavilla: esimerkiksi Euroopan digitaalisten innovaatiohubien (EDIH) kanssa yhteistyötä tehneistä yrityksistä 90 prosenttia on raportoinut digitaalisen kypsyytensä parantuneen \cite{Asikainen2025}.

Pk-yritykset hyötyvät tekoälystä monin tavoin; esimerkkejä ovat prosessien nopeutuminen, työaikasäästöt, työntekijätyytyväisyyden kasvu, parantuneet tuotteet ja palvelut, liiketoiminnan kasvattaminen ilman merkittäviä lisäresursseja, kustannussäästöt ja myynnin tehostuminen \cite{Jantti2025}. Vaikka tekoälyn käyttöönoton on arvioitu nostavan yritysten yleistä työvoiman tuottavuutta (kokonaistuottavuutta) tyypillisesti 2–3 prosenttiyksikköä vuodessa \cite{Thelle2024}, on jo tekoälykokeilujen varhaisessa vaiheessa havaittu huomattavaa tuottavuuden kehitystä \cite{Hajikhani2023}. Elisan arvion mukaan tekoälyn avulla voidaan poistaa jopa kolmannes pk-yritysten päivittäisistä rutiinitehtävistä \cite{Yrittajat2025}. Yli 60 prosenttia yrityksistä onkin ollut tyytyväisiä tekoälyinvestointiensa tuottamiin tuloksiin \cite{Jantti2025}. Tekoälyn koko potentiaali pk-yrityksissä piilee ihmisten ja koneiden välisessä saumattomassa yhteistyössä, jota usein kuvataan "tukiälynä" \cite{Seppala2023, HaagaHelia2023}. Tällöin yksittäisten tehtävien tuottavuusparannus voi olla huomattavasti suurempi; tuoreen, 106 kokeellista tutkimusta kattaneen meta-analyysin aineistoon perustuvan arvion mukaan ihmisen ja tekoälyn yhdistelmä oli tehtävissä keskimäärin noin 12 prosenttia tehokkaampi kuin ihminen yksin \cite{Vaccaro2024}.

Generatiivisen tekoälyn taloudellinen arvo syntyy pääosin asiakaspalvelun toiminnoista, markkinoinnista, myynnistä, tutkimus- ja kehitystyöstä (T\&K) sekä tuotekehityksestä \cite{KauttonenAsikainen2024}. Pk-yritykset aloittavat generatiivisen tekoälyn käyttöönoton usein sisäisten prosessien tehostamisesta, tavoitteenaan vapauttaa aikaa rutiinitehtävistä \cite{Jantti2025, Yrittajat2025} ja siirtyä osaamisen karttuessa kehittämään uusia, tekoälyä hyödyntäviä tuotteita tai palveluita \cite{Jantti2025}. Yleisimpiä käyttötarkoituksia ovat ideointi, kielikäännökset, viestintä, markkinointi ja myynnin kasvattaminen \cite{Ohlsbom2025, Yrittajat2025}. Markkinoinnin ammattilaiset hyödyntävät generatiivista tekoälyä esimerkiksi aivoriihiin, luovuuden edistämiseen, tekstien käsittelyyn ja editointiin sekä verkkosivujen A/B-testauksen nopeuttamiseen \cite{Ilonen2025}. Myös asiakaspalvelun älykkäät chatbotit ja virtuaaliassistentit, jotka pystyvät tarjoamaan personoidumpaa palvelua yrityksen omalla datalla, ovat yleistyneet \cite{KauttonenAsikainen2024, Rezazadeh2025}.

Sovellusesimerkkejä tekoälyn hyödyntämisestä eri toimialoilla ja tehtävissä ovat:
\begin{itemize}
    \item \textbf{Asiakaspalvelu ja kommunikaatio:} Generatiiviset tekoälybotit voivat automatisoida rutiinitehtäviä, kuten asiakastukea ja tiedusteluja, vähentäen ihmistyövoiman tarvetta ja tuoden kustannussäästöjä \cite{Heikkila2025}. Asiakaspalvelussa generatiivisen tekoälyn hyödyntäminen on kasvattanut käsiteltyjen tapausten määrää tunnissa keskimäärin 14 prosenttia, ja aloittelevien työntekijöiden tehokkuus on kasvanut jopa 34 prosenttia \cite{Krapi2024}.
    \item \textbf{Sisällöntuotanto ja markkinointi:} Pk-yritykset hyödyntävät tekoälyä markkinoinnissa ideointiin, tutkimukseen, sisällöntuotantoon ja asiakkaiden segmentoimiseen \cite{Heikkila2025, Yrittajat2025}. Parhaimmillaan generatiivisen tekoälyn on todettu jopa kolminkertaistavan sisällöntuotannon tehokkuuden \cite{Rezazadeh2025}. Markkinoinnin ohella myös myyntityössä tekoäly voi auttaa optimoimaan ajankäyttöä, analysoimaan kampanjoita ja hallitsemaan asiakasriskejä \cite{Lahtinen2024}.
    \item \textbf{Tietotyö ja hallinto:} Generatiiviset tekoälysovellukset ovat yleistyneet sisäisinä työkaluina parantaen tuottavuutta ja tehokkuutta tietotyössä, minkä on arvioitu säästävän jopa tunnin päivässä \cite{BusinessFinland2025}. Taloushallinnon asiantuntijat suhtautuvat generatiivisen tekoälyn sovelluksiin uteliaasti ja näkevät ne asiantuntijan assistentteina, jotka tarjoavat tietoa ja vinkkejä ohjelmistojen parempaan käyttöön ja tehtävien suorittamiseen \cite{Kuuva2024}.
    \item \textbf{Teollisuus ja tuotanto:} Valmistavassa teollisuudessa tekoälyä voidaan hyödyntää laajasti esimerkiksi teollisessa tutkimuksessa, tuotesuunnittelussa, laadunvalvonnassa, prosessien ohjauksessa, toimitusketjun hallinnassa ja ennakoivassa huollossa, mikä voi parantaa merkittävästi tuottavuutta ja vähentää tuotannon virheprosentteja sekä materiaalin tarvetta \cite{OECD2025, OECD2024}. Esimerkiksi Pirkanmaan pk-yrityksissä tekoälyä on käytetty valmistusprosessin optimointiin ja toiminnanohjaukseen, ja robotiikan alalla generatiiviset tekoälymallit voivat tehostaa robottien älykkyyttä sekä edistää ihmisen ja robotin välistä vuorovaikutusta \cite{Nieminen2023, OECD2024}.
    \item \textbf{Logistiikka:} Tekoälyalgoritmeja hyödynnetään kysynnän ennustamiseen, reittien optimointiin, työvoiman hallintaan ja kuorma-autojen täyttöasteen parantamiseen \cite{Ruusunen2025}.
    \item \textbf{Tuotekehitys:} Generatiivinen tekoäly voi tukea minimikelpoisen tuotteen (MVP) kehitysprosessia luomalla prototyyppejä kustannustehokkaasti ja tehokkaasti \cite{Rezazadeh2025}. Tekoäly mahdollistaa myös kokonaan uusien palveluiden ja toiminnallisuuksien kehittämisen \cite{BusinessFinland2025}.
\end{itemize}

Yleisesti ottaen tekoälyratkaisut edistävät yritysten digitaalista ja vihreää siirtymää, sillä ne voivat pienentää hiilijalanjälkeä ja optimoida esimerkiksi energiankulutusta \cite{Jantti2025, Chaudhuri2022}. Höytyjen tavoitteluun pyritään ohjaamaan myös kansallisilla TKI-panostuksilla. Business Finland tukee pk-yritysten generatiivisen tekoälyn hankkeita, jotka tähtäävät tuotteiden, palveluiden tai liiketoimintamallien merkittävään uudistamiseen tai kokonaan uusien kehittämiseen \cite{BusinessFinland2024}.